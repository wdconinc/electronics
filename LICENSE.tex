\documentclass[]{article}
\usepackage{lmodern}
\usepackage{amssymb,amsmath}
\usepackage{ifxetex,ifluatex}
\usepackage{fixltx2e} % provides \textsubscript
\ifnum 0\ifxetex 1\fi\ifluatex 1\fi=0 % if pdftex
  \usepackage[T1]{fontenc}
  \usepackage[utf8]{inputenc}
\else % if luatex or xelatex
  \ifxetex
    \usepackage{mathspec}
  \else
    \usepackage{fontspec}
  \fi
  \defaultfontfeatures{Ligatures=TeX,Scale=MatchLowercase}
\fi
% use upquote if available, for straight quotes in verbatim environments
\IfFileExists{upquote.sty}{\usepackage{upquote}}{}
% use microtype if available
\IfFileExists{microtype.sty}{%
\usepackage{microtype}
\UseMicrotypeSet[protrusion]{basicmath} % disable protrusion for tt fonts
}{}
\usepackage[letterpaper,margin=1in]{geometry}
\usepackage{hyperref}
\hypersetup{unicode=true,
            pdfborder={0 0 0},
            breaklinks=true}
\urlstyle{same}  % don't use monospace font for urls
\IfFileExists{parskip.sty}{%
\usepackage{parskip}
}{% else
\setlength{\parindent}{0pt}
\setlength{\parskip}{6pt plus 2pt minus 1pt}
}
\setlength{\emergencystretch}{3em}  % prevent overfull lines
\providecommand{\tightlist}{%
  \setlength{\itemsep}{0pt}\setlength{\parskip}{0pt}}
\setcounter{secnumdepth}{0}
% Redefines (sub)paragraphs to behave more like sections
\ifx\paragraph\undefined\else
\let\oldparagraph\paragraph
\renewcommand{\paragraph}[1]{\oldparagraph{#1}\mbox{}}
\fi
\ifx\subparagraph\undefined\else
\let\oldsubparagraph\subparagraph
\renewcommand{\subparagraph}[1]{\oldsubparagraph{#1}\mbox{}}
\fi

\date{}

\begin{document}

\subsection{Creative Commons Attribution-ShareAlike 4.0 International
(CC BY-SA
4.0)}\label{creative-commons-attribution-sharealike-4.0-international-cc-by-sa-4.0}

\paragraph{This is a human-readable summary of (and not a substitute
for) the
license.}\label{this-is-a-human-readable-summary-of-and-not-a-substitute-for-the-license.}

\paragraph{You are free to:}\label{you-are-free-to}

\textbf{Share} - copy and redistribute the material in any medium or
format

\textbf{Adapt} - remix, transform, and build upon the material

for any purpose, even commercially.

The licensor cannot revoke these freedoms as long as you follow the
license terms.

\paragraph{Under the following terms:}\label{under-the-following-terms}

\textbf{Attribution} - You must give \textbf{appropriate credit},
provide a link to the license, and \textbf{indicate if changes were
made}. You may do so in any reasonable manner, but not in any way that
suggests the licensor endorses you or your use.

\textbf{ShareAlike} - If you remix, transform, or build upon the
material, you must distribute your contributions under the \textbf{same
license} as the original.

\textbf{No additional restrictions} - You may not apply legal terms or
\textbf{technological measures} that legally restrict others from doing
anything the license permits.

\paragraph{Notices:}\label{notices}

You do not have to comply with the license for elements of the material
in the public domain or where your use is permitted by an applicable
\textbf{exception or limitation}.

No warranties are given. The license may not give you all of the
permissions necessary for your intended use. For example, other rights
such as \textbf{publicity, privacy, or moral rights} may limit how you
use the material.

\end{document}
