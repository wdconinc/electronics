\documentclass{article}

\usepackage{booktabs}
\usepackage{longtable}

\title{Electronics: Course Calendar}
\author{}
\date{Spring 2015}

\begin{document}
\maketitle


\begin{longtable}[c]{ll}
Date & Topic/readings
\endhead
\toprule
January 23 & ``Basic Electronics'' 1.1-1.3, 1.6 (skip 1.2.8) \\
(*) & Lab 1: ``Voltage, Current, Resistance, and Power'' \\
January 30 & ``Basic Electronics'' 1.3.4, 1.5-1.7 \\
(**) & Lab 2: ``Kirchhoff, Norton, Th\'{e}venin'' \\
February 6 & ``Basic Electronics'' 1.2.8, 2.1-2.4, 2.6 until eqn (2.42) \\
(**) & Lab 3: ``AC Signals, Complex Impedance, and Phase'' \\
February 13 & ``Basic Electronics'' 2.5-2.9, 3.1-3.4 \\
(**) & Lab 4: ``Passive Filters'' (``Basic Electronics'' 3.7-3.8) \\
February 20 & ``Basic Electronics'' 4.1-4.6 (4.1-4.4 as background) \\
(*) & Lab 5: ``Diodes'' \\
February 27 & Review of chapters 1-4 \\
(*) & Midterm \& Lab 6: ``5Spice and Eagle'' \\
March 6 & ``Basic Electronics'' 5.1-5.3.1 \\
March 13 & Spring break \\
(**) & Lab 7: ``Introduction to Transistors'' \\
March 20 & ``Basic Electronics'' 6.1-6.3 \\
(*) & Lab 8: ``Basic Op-Amps Circuits'' \\
March 27 & ``Basic Electronics'' 6.4-6.5 \\
(**) & Lab 9: ``Op-Amps and Detectors'' \\
April 3 & ``Basic Electronics'' 6.6 \\
(***) & Lab 10: ``PID Feedback and Control'' \\
April 10 & ``Basic Electronics'' 6.8 \\
(***) & Lab 11: ``Comparators and Oscillators'' \\
April 17 & ``Basic Electronics'' 7.1-7.5 \\
(*) & Lab 12: ``Introduction to Digital Electronics'' \\
April 24 & ``Basic Electronics'' 7.6-7.8 \\
(**) & Lab 13: ``Digital Circuits'' \\
May 1 & Review \\          
May 4 & Final \\
\bottomrule
\end{longtable}                

\noindent Stars are a subjective indicator of how difficult/long the lab typically is. Insufficient preparation for the labs with 3 stars could mean a late night in Small Hall\ldots{}

\end{document}
