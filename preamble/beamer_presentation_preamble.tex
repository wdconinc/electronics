%\includeonlyframes{draft}
% This file is a solution template for:

% below to make handout 
%\usepackage{pgfpages}
%\pgfpagesuselayout{4 on 1}[letterpaper,landscape, border shrink=5mm]

% This file is a solution template for:

% - Talk at a conference/colloquium.
% - Talk length is about 20min.
% - Style is ornate.

\usepackage{standalone}
\usepackage{import}

\usepackage{bookmark}

\usepackage{pgf}
\usepackage{tikz}
\usepackage{bodegraph}
\usepackage{circuitikz}
\usetikzlibrary{calc,arrows,shapes,trees,fadings,positioning,intersections}

\tikzstyle{block} = [draw, fill=blue!20, rectangle, 
    minimum height=3em, minimum width=6em]
\tikzstyle{sum} = [draw, fill=blue!20, circle, node distance=1cm]
\tikzstyle{input} = [coordinate]
\tikzstyle{output} = [coordinate]
\tikzstyle{pinstyle} = [pin edge={to-,thin,black}]

% Copyright 2004 by Till Tantau <tantau..
%
% In principle, this file can be redistributed and/or modified under
% the terms of the GNU Public License, version 2.
%
% However, this file is supposed to be a template to be modified
% for your own needs. For this reason, if you use this file as a
% template and not specifically distribute it as part of a another
% package/program, I grant the extra permission to freely copy and
% modify this file as you see fit and even to delete this copyright
% notice. 

\mode<presentation>
{
  \usetheme{Madrid}  % for short upt to 20min presentation
  \setbeamercovered{transparent}

  \setbeamertemplate{footline}%
  {%
    \leavevmode%
    \hbox{%
    \begin{beamercolorbox}[wd=.5\paperwidth,ht=2.25ex,dp=1ex,left]{}%
      \hspace*{1ex} \insertlogo
    \end{beamercolorbox}%
    \begin{beamercolorbox}[wd=.5\paperwidth,ht=2.25ex,dp=1ex,right]{}%
      \insertframenumber{} \hspace*{1ex}
    \end{beamercolorbox}}%
    \vskip0pt%
  }

  % No navigation symbols
  \setbeamertemplate{navigation symbols}{}
}


\usepackage[english]{babel}
% or whatever

\usepackage[latin1]{inputenc}
% or whatever

\usepackage{times}
\usepackage[T1]{fontenc}
% Or whatever. Note that the encoding and the font should match. If T1
% does not look nice, try deleting the line with the fontenc.

\usepackage{listings}
\usepackage{amsmath}



%\setbeamercovered{transparent}
\setbeamercovered{invisible}



%\subtitle
%{Include Only If Paper Has a Subtitle}

\author[W. Deconinck] % (optional, use only with lots of authors)
{
  Wouter Deconinck
}


%{Wouter Deconinck\inst{1} \and S.~Another\inst{1}}
% - Give the names in the same order as the appear in the paper.
% - Use the \inst{?} command only if the authors have different
%   affiliation.

% If you have a file called "university-logo-filename.xxx", where xxx
% is a graphic format that can be processed by latex or pdflatex,
% resp., then you can add a logo as 

%remember extension must be ommited
\pgfdeclareimage[height=1cm]{cc-by-sa}{styles/CC-BY-SA_icon.png}
%\logo{\pgfuseimage{cc-by-sa}}

\institute[CC BY-SA] % (optional, but mostly needed)
{
  \pgfuseimage{cc-by-sa}
}
% - Use the \inst command only if there are several affiliations.
% - Keep it simple, no one is interested in your street address.

% If you wish to uncover everything in a step-wise fashion, uncomment
% the following command: 
%\beamerdefaultoverlayspecification{<+->}

%
\providecommand{\fixme}[1]{%
  { \em ---! FixMe #1 !---}
}
%
\providecommand{\hide}[1]{%
}
%
\providecommand{\mat}[1]{% matlab code
{\color{blue}\texttt{#1}}%
}
%
\providecommand{\mstr}[1]{% matlab code string
{\color{magenta}#1}%
}

% {{{ listing color schem definitions
\definecolor{listcommentcolor}{rgb}{0.0,0.4,0} %dark green
%\definecolor{listbg}{rgb}{0.67,0.90,0.64} %pale green
\definecolor{listbg}{rgb}{0.90,0.90,0.90} %kight grey
\lstset{backgroundcolor=\color{listbg}}
\lstset{commentstyle=\color{listcommentcolor}}
\lstset{keywordstyle=\color{blue}}
\lstset{stringstyle=\color{magenta}}
\lstset{language=matlab}
\lstset{tabsize=2}
\lstset{showstringspaces=false}
\lstset{basicstyle = \ttfamily}
% }}}

